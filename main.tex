\documentclass[twocolumn]{article}

\usepackage{amsmath}

\usepackage{bm}

\usepackage{physics}

\usepackage{graphicx}
\usepackage{epstopdf}
\graphicspath{{img/}}

\everymath{\displaystyle}

\begin{document}

\title{Restricted Boltzmann Machines}
\author{Giancarlo Fissore}
\date{May 2017}
\maketitle

\begin{abstract}
  Abstract text
\end{abstract}

\section{Introduction}

\section{Training}
A Restricted Boltzmann Machine (RBM) is a model for neural networks which basically consists in a bipartite graph with a layer of hidden units \(h_i\) and a layer of visible units \(v_i\). The units in one layer are not connected among them but are connected to all the units in the other layer, as shown in fig. x. We restrict our treatment to the case of binary units \(h_i,v_i = 0,1\). Drawing a comparison between RBMs and spin models in statistical physics, we can define the following \textit{energy function}

\begin{equation}
E(\textbf{h},\textbf{v}) = - \sum_i a_i v_i - \sum_j b_j h_j - \sum_{i,j} v_i w_{ij} h_j
\end{equation}

where \(a_i\) and \(b_i\) are \textit{external fields} acting respectively on the visible and hidden units. The probability of a certain configuration is then given by the Boltzmann measure (taking \(\beta = 1\))

\begin{equation}
P(\textbf{h},\textbf{v}) = \frac{e^{-E(\textbf{h},\textbf{v})}}{Z}
\end{equation}

where \(Z = \sum_{\textbf{h},\textbf{v}} e^{-E(\textbf{h},\textbf{v})}\) is the \textit{partition function}. The probabilities of activations for visible and hidden units can be simply computed to be

\begin{align}
P(v_i = 1 | \textbf{h}) &  = \frac{1}{1+e^{-a_i - \sum_{j} w_{ij} h_j}} \nonumber \\
& = \sigma \left(a_i + \sum_{j} w_{ij} h_j \right) \\
P(h_j = 1 | \textbf{v}) & = \frac{1}{1+e^{b_j + \sum_i w_{ij} v_i}} \nonumber \\
& = \sigma \left(b_j + \sum_i w_{ij} v_i \right)
\end{align}

We see that the usual neuron activation function (the sigmoid function, which we indicated by \(\sigma\)) comes out naturally by choosing binary units. Moreover, we can now understand the importance of the external fields: when negative, they drive the activation of a certain unit by setting a threshold that has to be overcome by the opposite layer through the couplings \(w_{ij}\).

What we are interested in is the probability for the visible units, which is the layer we use to represent data. It can be easily defined as

\begin{align}
P(\textbf{v}) &  = \sum_{\textbf{h}} P(\textbf{h},\textbf{v}) \nonumber \\
& = \frac{e^{-F(\textbf{v})}}{Z}, \quad Z = \sum_{\textbf{v}} e^{-F(\textbf{v})}
\end{align}

where, further exploiting the parallel with spin models, we have introduced the \textit{free energy}

\begin{align}
F(\textbf{v}) & = -\log \sum_{\textbf{h}} e^{-E(\textbf{h},\textbf{v})} \nonumber \\
& = -\sum_i a_i v_i -\sum_j \log \sum_{h_j} e^{h_j \left( b_j + \sum_i w_{ij} v_i \right)} \nonumber \\
& = -\sum_i a_i v_i -\sum_j \log \left( 1 +  e^{\left( b_j + \sum_i w_{ij} v_i \right)} \right)
\end{align}

To use the RBM as a generative model, we want to maximize \(P(\textbf{v}\)) (and thus minimize the free energy) for the samples belonging to the training set. Once this is done, we can sample the equilibrium configurations of the RBM to obtain samples which are generated according to the probability distribution of the training data. The algorithms used to train and sample RBMs are usually based on Monte Carlo methods. In particular, the most effective algorithms for training are \textit{k-steps contrastive divergence} (CDk) \cite{Hinton_CD}  and \textit{persistent contrastive divergence} (PCD) \cite{PCD}. 

\section{Spectral analysis}
The structure of the samples that a RBM  is able to generate must be in some way encoded into the external fields \(\textbf{a},\textbf{b}\) and the weights matrix \(\textbf{W}\), as these are the set of quantities which constitute the RBM itself. For what concerns the visible field values \(a_i\), these are used to make sure that the visible unit \(i\) is activated with a probability \(p_i\) given by the proportion of training samples in which unit \(i\) is active (i.e. where its value is 1). No learning is needed to make the field \(\textbf{a}\) encode this information, it is sufficient to use the initialization rule \(a_i = \log[p_i/(1-p_i)]\) as reported in \cite{Hinton_guide}. To understand how the structure of the data is encoded into the weights matrix, instead, we monitored the singular value decomposition (SVD) of \textbf{W} during the training, given by

\begin{equation}
\textbf{W} = \textbf{U\(\Sigma V^T \) } 
\end{equation}

where the columns of \textbf{U} and \textbf{V} are respectively the left and right singular vectors and \textbf{\(\Sigma\)} is the diagonal matrix of the singular values in decreasing order. 
In the context of a RBM, we can better specify the role of the SVD matrices:

\begin{itemize}
\item \textbf{U} encodes the singular vectors related to the visible layer and can therefore be visualized in the pixel space
\item  \textbf{V} is related to the hidden layer and it is a square orthogonal matrix that can be interpreted as a rotation. 
\item The singular values \( \{ {\sigma}_j \} \) contained in \textbf{\(\Sigma\)} can be thought of as scaling factors whose action is to weight the singular vectors composing \textbf{W}.
\end{itemize}

Given the above characteristics we focused our attention on \textbf{\(\Sigma\)} and \textbf{U}, tracking the distribution of the singular values and looking at the corresponding left singular vectors during the training.

\subsection{Distribution of the singular values}
The weights matrix \(\textbf{W}\) is initialized as a gaussian random matrix, whose singular values distribution is known to be given by the Marchenko-Pastur law. Fig. x shows the agreement between the empirical distribution and the theoretical distribution. In particular, we note how all \(\sigma_j\) have values below the threshold set by the Marchenko-Pastur law, forming a \textit{bulk} of singular values.

Starting with the training we see that many singular values increase in value and overcome the threshold for a gaussian random matrix; these are \textit{outliers} leaving the bulk, shown in fig. x. During the first epochs of the training this process is very fast and many \(\sigma_j\) are easily extracted from the bulk, growing of many orders of magnitude. The bulk is instead shrinked to low values, meaning that the \(\sigma_j\) which do not overcome the threshold decrease in magnitude. Going on with the training this process slows down but it does not stop: outliers keep growing slowly and the bulk keep shrinking to approach a spike around zero. The evolution of the \(\sigma_j\) distribution is shown in fig. x. After a long training the singular values \(\sigma_j\) are separated into two categories: a concentrated set of almost-null singular values and a "expanded" set of outliers above threshold, as shown in fig. x. 

This process highlights a twofold effect: the singular vectors corresponding to the above threshold \(\sigma_j\) are strengthen by the training, while the

 as shown in fig. x. The evolution of this distribution is shown in fig. x; as the training goes on, more and more singular values acquire importance and overcome the threshold of the distribution for a random matrix while the remaining singular values decrease to almost null values.


\begin{thebibliography}{9}

\bibitem{go}
D. Silver et al., "Mastering the game of Go with deep neural networks and tree search",
\textit{Nature}, 529, p. 484–489, 2016.

\bibitem{foundations}
F. Cucker, S. Smale, "On the mathematical foundations of learning",
\textit{Bull. Amer. Math. Soc.}, 39, p. 1-49, 2002.

\bibitem{hist1}
W. Krauth, M. Mezard, "Machine Learning algorithms with optimal stability in neural networks",
\textit{Journal of Physics A: Mathematical and General}, 20, L745–L752, 1987.

\bibitem{hist2}
J. J. Hopfield, "Neural networks and physical systems with emergent collective computational abilities"\textit{Proceedings of the National Academy of Sciences, 79}, 8, p. 2554–2558, 1982.

\bibitem{tap_train}
M. Gabri\'e, E. W. Tramel, F. Krzakala,
Training Restricted Boltzmann Machines via the Thouless-Anderson-Palmer Free Energy,
\textit{Advances in Neural Information Processing Systems (NIPS)}, 28, pages 640--648, 2015.

\bibitem{tap}
E. W. Tramel, M. Gabri\'e, A. Manoel, F. Caltagirone, F. Krzakala, \textit{arXiv:1702.03260}

\bibitem{monasson}
J. Tubiana, R. Monasson, "Emergence of Compositional Representations in Restricted Boltzmann Machines",
\textit{Phys. Rev. Lett. 118, 138301}, 2017.

\bibitem{Hinton_CD}
G. E. Hinton, “Training products of experts by minimizing Contrastive divergence,”
\textit{Neural computation}
, vol. 14, pp. 1771-1800, 2002.

\bibitem{PCD}
T. Tieleman, “Training restricted Boltzmann machines using approximations to the likelihood gradient,”
\textit{ICML}, Vol. 307, p. 7, 2008.

\bibitem{Hinton_guide}
G. E. Hinton, "A Practical Guide to Training Restricted Boltzmann Machines", \textit{Proceedings of Neural Networks: Tricks of the Trade (2nd ed.)}, pages 599-619, 2012. 

\bibitem{PCA}
I.T. Jolliffe, "Principal Component Analysis",
\textit{Springer-Verlag New York}, 2002.

\bibitem{SK}
D. Sherrington, S. Kirkpatrick,
"Solvable Model of a Spin-Glass",
\textit{Phys. Rev. Lett.}, Vol. 35, 1975.

\bibitem{ht_exp}
A. Georges, J. S. Yedidia,
"How to expand around mean-field theory using high-temperature expansions",
\textit{Journal of Physics A: Mathematical and General}, Volume 24, Number 9, 1991

\bibitem{TAP}
D. J. Thouless , P. W. Anderson, R. G. Palmer,
"Solution of 'Solvable model of a spin glass'",
\textit{Philosophical Magazine}, Vol. 35:3, p. 593-601, 1977

\bibitem{conv}
E. Bolthausen, "An Iterative Construction of Solutions of the TAP Equations for the Sherrington–Kirkpatrick Model", \textit{Commun. Math. Phys.}, Vol. 325, pages 333-366, 2014.

\bibitem{mnist}
http://yann.lecun.com/exdb/mnist/

\bibitem{gibbs}
G. Casella and E. I. George, "Explaining the Gibbs Sampler",
\textit{The American Statistician},
Vol. 46, No. 3, pp. 167-174, 1992.

\bibitem{leroy}
C. Leroy made an investigation on TAP fixed points and free energy landscape while at Laboratoire de Recherche en Informatique, working in my same team.

\bibitem{Nishimori}
H. Nishimori, "Statistical Physics of Spin Glasses and Information Processing: An Introduction",
\textit{Oxford University Press}, 2001

\bibitem{MP_law}
Marchenko, V.A. and Pastur, L.A., "Distribution of Eigenvalues for Some Sets of Random Matrices", \textit{Sbornik: Mathematics}, Vol. 1, pages 457-483, 1967.

\bibitem{ganguli}
A. M. Saxe, J. L. McClelland, S. Ganguli, "Exact solutions to the nonlinear dynamics of learning in deep linear neural networks",
\textit{arXiv:1312.6120}, 2014.

\end{thebibliography}


\end{document}
